\documentclass[a4paper,10pt]{report}
\usepackage[utf8]{inputenc}
\usepackage{longtable}
\usepackage{xcolor}
\usepackage{listings}

% Title Page
\title{Identification of Unique Oligonucleotides: Users Guide}
\author{Mihaela Martis}


\begin{document}
\maketitle
\section{Getting started}
\label{sec:start}
The pipeline detects unique DNA oligonucleotides in the human genome using a combination of external tools and internal filtering steps. The identified
oligomers are selected based on the following criteria: uniqueness of the genomic location, melting temperature, GC content, and inability to form a
secondary structure. The unique oligomers are detected using the Jellyfish (version 2.2.3) software, while Vmatch (version 2.2.4) return their genome/chromosom
positions. Vmatch runs twice, first to determine if the selected oligos are unique over the entire length, second by allowing up to 5 mismatches. The second
run allows a much more restrictive filtering of the oligonucleotides. Nevertheless, the number of allowed mismatches should not exceed 5. Due to the intensive
search necessary to perform the task, the allocated core hours and time would be exceeded and the pipeline will fail.
\begin{lstlisting}[basicstyle=\tiny]
vmatch -q in.fa -l 60 -p -d -selfun mydb.so -showdesc 100 index > out.fa
vmatch -q in.fa -complete -p -d -h 9b -selfun mydb.so -showdesc 100 index > out.fa
\end{lstlisting}
The degree of free energy is calculated by the OligoArrayAux software (version 3.8). All oligomers with an GC-content and melting temperature below
or above the given intervals, and with a $\Delta$G (free energy) $<$ 0 are filtered out from the final data set. In addition the oligomers containing homopolymers
(e.x.~AAAAA or TTTTT or GGGG or CCCC) can be droped out, too. The melting temperature is calculated using the following formula:\\
\begin{center}
  $81.5 + (0.41 * GC) + (16.6 * log10(Na)) - (500/K) - (0.62 * F)$,\\
\end{center}
where
\begin{itemize}
 \item[]GC = \%GC-content
 \item[]Na = salt concentration
 \item[]F  = formamid concentration
 \item[]K  = k-mer size
\end{itemize}
The unique oligomers are uploaded to a SQLite database, which can be enquired using the same pipeline (see \ref{sec:howto}) or by using the SQLite Manager add-on for
the Firefox-browser (http://www.sqlabs.com/sqlitemanager.php). For each run the data will be saved in a table, which name will contain the following parameters: kmer
(\textcolor{blue}{-k}), minimal and maximal GC content (\textcolor{blue}{-g} and \textcolor{blue}{-G}), reference name (\textcolor{blue}{-n}), maximal overlap between
2 oligomers (\textcolor{blue}{-v}), the delta needed to increase or decrease the melting temperature, as well as the information if a homopolymer filtering has been 
done or not (\textcolor{blue}{-F}) (e.~x.~\textit{kmer60\_dtm10\_gcmin35\_gcmax80\_hpolno}). The table will be overwritten with each run, except one of the above mentioned
parameters are changed.

The pipeline automatically generates a log file for each individual chromosom to describe step by step the design process. The main interest of this files are the
explanations of why the design failed for a given step. This can help to figure out if a parameter was too stringent and need to be relaxed for a second run or if
other events occured, which caused the pipelinie to crash.

Furthermore, the files \textcolor{blue}{.bash\_profile} and \textcolor{blue}{.bashrc} located in the home directory need to be customized. This is necessary, because
Vmatch, OligoArrayAux and some Perl libraries required by the pipeline, could be installed only locally. Please contact me (mihaela.martis@bils.se) to get further
information.

\section{Options}
\label{sec:param}
This section describes the command line options to run the pipeline in order to identify unique oligomers.\\

\begin{longtable}{p{.25\textwidth}p{.75\textwidth}}
  -a, -{}-all & This option accepts only \textcolor{blue}{yes} or \textcolor{blue}{no} as arguments and it is mandatory. It specifies if the pipeline should perform only a database enquiry or the identification of unique k-mers.\\
  -b, -{}-begin\_pos & The start position of the region of interest. This value is necessary if unique oligomers from a certain region should be extracted from the database. This value is optional.\\
  -c, -{}-chromosom & The chromosom of the region of iterest (e.{}x.~\textcolor{blue}{1, 8, X, or Y}). This value is optional.\\
  -d, -{}-db\_name & The path and name of the SQLite database. If this option is not used, then the default path is equal to the working directory and the database name is \textcolor{blue}{unique\_oligonucleotides.db}\\
  -D, -{}-distance & The minimum distance between two unique oligomers. If this option is not used, then the default value is \textcolor{blue}{10}.\\
  -e, -{}-end\_pos & The stop position of the region of interest. This value is necessary if unique oligomers from a certain region should be extracted from the database. This value is optional.\\
  -f, -{}-formamid & The percentage of the formamid concentration. If this option is not used, then the default value is \textcolor{blue}{50}.\\
  -F, -{}-hpol\_flag & This option accepts only 'yes' or 'no' as arguments. 'yes' means that homopolymers were filtered in the table of interest, 'no' means there was no homopolymer filtering done. It is mandatory if you want to run only the database enquiry.\\
  -g, -{}-gc\_min & The minimum oligomer GC content in percentage. A positive integer between 0 and 100 is expected. If this option is not used, then the default value is \textcolor{blue}{35}.\\
  -G, -{}-gc\_max & The maximum oligomer GC content in percentage. A positive integer between 0 and 100 is expected. If this option is not used, then the default value is \textcolor{blue}{80}.\\
  -h, -{}-help & Show a summary of all options available for this pipeline and terminate the pipeline.\\
  -k, -{}-kmer & The length of the unique oligomers. This option is mandatory and should be a positive integer between 20 and 100.\\
  -m, -{}-mail & The email address to which the SLURM system should report if the jobs fail or complete. Please use quotation marks (' ') around the email address. [default: 'mihaela.martis$@$bils.se'].\\
  -M, -{}-mismatch & The number of allowed mismatches in the second Vmatch run. The number should be not higher than 5, due to runtime issues.\\
  -n, -{}-refname & The name of the reference genome. [default: human]\\
  -o, -{}-homopolymer & A list of homopolymers separated by semi-colon, which should not be contained in the oligomer sequences (e.{}x.~\textcolor{blue}{GGGG;CCCC;AAAAA;TTTTT}). If this option is not used, then the default value is \textcolor{blue}{na} (no homopolymer filtering).\\
  -p, -{}-project & Uppmax project name. If this option is not used, then the default value is \textcolor{blue}{b2015233}.\\
  -r, -{}-reference & The path to the individual human chromosomes. If this option is not used, then the default is \textcolor{blue}{/sw/data/uppnex/reference/Homo\_sapiens/hg19/chromosomes}\\
  -s, -{}-salt & The salt concentration [Na+] in M. If this option is not used, then the default value is \textcolor{blue}{0.42}.\\
  -S, -{}-ss\_temp & The threshold temperature for the calculation of the free energy of the secondary structure. If this option is not used, then the default value is \textcolor{blue}{65}.\\
  -t, -{}-tm\_delta & The degree of which the average melting temperature should be increased and decreased to form an interval. A positive integer is expected. If this option is not used, then the default value is \textcolor{blue}{10}.\\
  -u, -{}-user & UPPMAX user id.\\
  -v, -{}-maxoverlap & The maximal overlap between 2 oligomers in percentage.\\
  -w, -{}-workdir & The absolute path to the working directory. If this option is not used, then the default working directory is: \textcolor{blue}{/proj/b2015233/nobackup/uniqueOligoPipeline}.\\
  -W, -{}-wgs & The absolute path to the whole genome sequence of interest. The default directory is: \textcolor{blue}{/sw/data/uppnex/igenomes/Homo\_sapiens/Ensembl/GRCh37/Sequence/Whole GenomeFasta/genome.fa}.\\
\end{longtable}

\section{How to}
\label{sec:howto}
The pipeline needs to be run from the command line by using various options described above (see \ref{sec:param}). There are three ways to run the pipeline:
  \begin{itemize}
    \item only to enquiry the database for unique oligomers from a certain region on a given chromosome. This presupposes that the unique kmers were already identified
    and stored in the database. In this case the options \textcolor{blue}{-a}, \textcolor{blue}{-k}, \textcolor{blue}{-c}, \textcolor{blue}{-b}, \textcolor{blue}{-e}, and
    \textcolor{blue}{-F} are mandatory:
    \item[] \textcolor{blue}{perl uniqueDnaOligonucleotidesCatcher.pl -a no -k 60 -c 10 -b 15000 -e 30000 -F yes}
    \item[] \textcolor{blue}{perl uniqueDnaOligonucleotidesCatcher.pl -a no -k 60 -c 10 -b 15000 -e 30000 -F no}
    \item[] The \textcolor{blue}{-F} option specifies whether a homopolymer filtering has been performed or not.
    \item to identify unique oligomers for a given k-mer size without database enquiry. In this case only the options \textcolor{blue}{-a} and \textcolor{blue}{-k} are mandatory:
    \item[] \textcolor{blue}{perl uniqueDnaOligonucleotidesCatcher.pl -a yes -k 60}
    \item to identify unique oligomers for a given k-mer size and enquiry the database for a certain region:
    \item[] \textcolor{blue}{perl uniqueDnaOligonucleotidesCatcher.pl -a yes -k 60 -c 10 -b 15000 -e 30000}
    \item[] In this case the \textcolor{blue}{-F} option is not required. Whether a homopolymer filtering occured or not is deduced automatically.
  \end{itemize}
If oligomers containing homopolymers should be excluded from the final data set, too, then the following cue can be used:\\\\
\textcolor{blue}{perl uniqueDnaOligonucleotidesCatcher.pl -a yes -k 60 -o 'AAAAA;TTTTT;CCCC;GGGG'}.\\
Here, the quotation marks (\textcolor{blue}{' '}) are very important and should not be forgotten.
\end{document}          
